% arara: pdflatex: { shell: yes } until !found('log', '\\(?(R|r)e\\)?run (to get|LaTeX)')
\documentclass[a4paper,
DIV=13,
12pt,
BCOR=10mm,
department=FakEI,
%lucida,
%KeepRoman,
twoside,
parskip=half,
automark,
%headsepline,
]{OTHRartcl}

\usepackage[utf8]{inputenc}
\usepackage{breakcites}

%\usepackage[english]{babel}
\usepackage[ngerman]{babel}

\date{\today}

\pagestyle{headings}

\title{Ist Transhumanismus Fortschritt oder Dystopie?}
\usepackage{url}
\author{Marcel Ott}

\documenttype{Ethik}

\begin{document}
\maketitle

\section*{Einleitung}
Videoausschnitt abspielen~\cite{HughHerrYoutube} Die gezeigte Person heißt Jim Ewing und der Videoausschnitt stammt aus einem Tedtalk von Hugh Herr über agonist-antagonist myoneural interface Prothesen aus dem Jahr 2018 mit denen
es möglich ist ein Nervenfeedback bei Bewegung zu erhalten und somit fühlt sich die Bewegung an als würde man seine eigene Gliedmaße bewegen. Jim war vor seinem Unfall ein leidenschaftlicher Kletterer und mit diesem System
ist es ihm möglich so feine Bewegungsmuster auszuführen , dass es ihm jetzt wieder möglich ist Steilwände hochzuklettern.%vllt Videoausschnitt wo er hochklettert im hintergrund einblenden oder nur Foto?
Der Clip verdeutlich einerseits das Transhumanismus heute schon intensiv erforscht wird und andererseits herrlich wie schnell sich Menschen an transhumanistische Mittel gewöhnen und akzeptieren. Daher fragen wir uns,
ob der Transhumanismus einen Fortschritt oder gar eine Dystopie darstellt.

\section*{Definitionen bzw. Abgrenzung der Begriffe}
Doch was ist eigentlich Transhumanismus?
Der Transhumanismus beschäftigt sich mit dem Menschen und die Ausschöpfung bzw. Weiterentwicklung seiner natürlichen Grenzen mithilfe von Technik und Wissenschaft~\cite{Merzlyakov2022},
wobei der Mensch als solches weiter beibehalten wird. Hingegen sieht der Posthumanimus den Menschen als Sackgasse an, welche es zu überwinden gilt. Hierbei werden die binären Gegensätze
zwischen Mensch und nicht-Mensch aufgehoben.~\cite{Merzlyakov2022} Der Cyborg wird oft als nächste Stufe der Evolution betitelt. Jedoch sind die Grenzen fließend und die Begriffe werden
oft synonm verwendet, was aber von einigen Forschern kritisiert wird, da es wie gezeigt fundamentale Unterschiede gibt~\cite{Merzlyakov2022} Der Begriff des Cyborgs beschreibt ein
integriertes System aus menschlichen und maschinellen Teilen~\cite{warwick2000cyborg} und wurde bereits 1960 verwendet.~\cite{clynes1960cyborgs}.

\section*{Was ist normal?}
Anetta Breczko meint die Überwachung biotechnologischer Möglichkeiten erfordert zweifellos in erster Linie eine Unterscheidung zwischen „therapeutischen“ und „Verbesserungs“-Aktivitäten~\cite{breczko2021human}
Doch um diese Unterscheidung treffen zu können muss man ersteinmal entscheiden, was normal ist und somit als therapeutisch zu sehen ist. Inituitive scheint diese Fragestellung trivial zu sein
man nimmt den statischen Durchschnitt, was in manchen Kontexten Sinn macht oder wie in der Vergangenheit, z. B. bei der Sklaverei in der POC als minderwertig angesehen wurden, oft Geschehen die herrschende Klasse
bestimmt was normal ist. Das dies in der transhumanistischen Betrachtung wenig Sinn macht zeigt der Fall des Cholea-Implantates, welcher später genauer erwähnt wird eindrucksvoll. Ulrike Schildmann beschreibt in einem 
Aritkel die Normalität als sehr indiviuell und vom Selbst oder der umgebenden Gruppe bestimmt.\cite{schildmann1999normal} Einige Aktivisten, wie Raúl Aguayo-Krauthausen, fordern
eine Behinderung nicht als Markel welchen es zu beseitigen geht sondern als Eigenschaft, wie die Augenfarbe anzusehen.~\cite{aguayo2023inklusion} Ein Slogan entnommen aus den ethischen Grundaussagen der Lebenshilfe
für Menschen mit geistiger Behinderung besagt: ``Es ist normal, verschieden zu sein.''~\cite{lebenshilfeFlyer}.

\section*{Einige Chancen des Transhumanismus}
Im Laufe des Vortrags werden weit mehr und detailierter die Chancen folgen, doch will ich hier einen kleinen Rundumschlag geben, damit man ein Grundverständnis erlangt, von was gesprochen wird.
Man wird in Zukunft psychische Leiden, wie Angststörungen, Depressionen oder PTBS mittels DBS ~\cite{perlmutter2006deep} oder TMS non invasis heilen können.~\cite{hallett2007transcranial} heilen können. Es wird wahrscheinlich sogar möglich sein Charaktereigenschaften wie Extraversion herbeizuführen. Ein anderer Anwendungszweck ist die verbesserte Leistungsfähigkeit, wobei z.B. mittels TMS
Prüfungsleistungen~\cite{luber2014enhancement} verbessert werden können. Auch eine Steigerung der Ausdauer, wie Kurzweil es mit seinen Nanobots im Blut postuliert~\cite{kurzweil2005singularity} scheint möglich. Diese Ansätze entsprechen dem kapitalistischen Grundgedanken
unserer Gesellschaft, wobei stetige Verbesserung eine Voraussetzung für Wachstum ist. Nun stellt sich die Frage, wieso bei Produktsmitteln aufhören? Doch diese Vorteile kommen nicht ohne Preis und bringen viele teils gravierende
ethische Fragestellungen mit sich, mit denen der Nicolas weiter macht.

\subsection*{Selbstbestimmung des Indiviuum}

\begin{itemize}
    \item Grundätzlich hat jeder das Recht auf freie Entfaltung solange er nicht Rechte anderer oder bestehendes Recht verletzt\cite{fur1996grundgesetz}. 
    \item Die eigenen Identität sollte von Menschen selbst gewählt werden. 
    \item Dazu kommt das Argument natürlich bleiben zu wollen~\cite{lee2016cochlear}.
    \item Ein weiterer Aspekt ist die Frage ob Menschen überhaupt frei entscheiden können, wenn die Mehrheit der Menschen verbessert ist können Menschen die sich nicht verbessern wollen 
    nicht mehr am Altagsleben teilnehmen können weil sie keinen Jobs nachgehen können. Ein Beispiel dafür ist Profi Bodybulding, wenn man nicht bereit ist Steroide zu nehmen kann man
    den Sport nicht machen. Ein konkretes Beispiel wie Profi Bodybuilding und der Einsatz von Steroiden illustriert diesen Punkt. Wenn die Mehrheit der Athleten auf leistungssteigernde 
    Substanzen zurückgreift, könnten jene, die sich gegen diese Verbesserungen entscheiden, möglicherweise Schwierigkeiten haben, auf dem gleichen Wettbewerbsniveau zu bleiben.
\end{itemize}

\subsection*{Entscheidungen Treffen für andere:}
Da es von außen schwer zu sagen was Leiden ist es Schwer Entscheidungen für andere zu Treffen~\cite{plavsienkova2021healthy}. Jeder muss für sich selbst abwägen ob er Nebenwirkungen 
in Kauf nehmen will oder nicht. 

Eine Andere sichtweise ist das die Gesellschaft die Konsequenzen Tragen muss wenn sich Menschen sich gegen eine Verbesserung Entscheiden. Mögliche Konsequenzen wären dabei höhere 
Gesundheitskosten, und weiter kosten die durch entstehen die Menschen in das Altagsleben einzubinden. => Daraus könnte eine Pflicht zur Verbesserung erfolgen.

Die Frage, welche sich am meisten stellt ist ob bei Menschen, welche nicht selbstbestimmt entscheiden können über den Kopf hinweg entschieden werden  und wenn ja wie. Das wohl 
prägnanteste Beispiel wäre das Locked-in-Syndrom, bei dem der Hirnstamm beschädigt ist und die betroffenen somit normal denken und fühlen können, jedoch sich nicht bewegen oder 
sprechen können.~\cite{das2022locked}

Dazu kommt das man den Menschen wenn für jemanden anderen eine Entscheidung gegen eine Verbesserung trifft die Betroffene Personen später  möglicherweise Einschränkungen in der 
Teilhabe am Leben in der Gemeinschaft haben wird. Besonders Kritisch wird es wenn es um Eltern geht die eine Entscheidung für ihr Kind treffen sollen, denn man könnte Argumentieren 
das es sich um eine Gefährdung des Kindeswohls handelt wenn Eltern sich gegen eine Verbesserung entscheiden.
Diese Frage würde tatsächlich schon vor Gericht diskutiert. 2017 Standen gehörlose Eltern vor der Entscheidung ihrem Kind ein CI-Implantat zu implantieren oder nicht. Ein Cochlea 
Implantat ist dabei ein Implantat das in das Ohr implantieren wird und Gehörlosen Menschen somit das Hören ermöglichen kann. Die Eltern Entschieden sich gegen das Implantat
woraufhin die HNO-Klinik des Städtischen Klinikums Braunschweig die Weigerung der Eltern als Gefährdung des Kinderwohls und leitete ein Kinderschutzverfahren ein. 
Die Klinik argumentierte, dass die Ablehnung einer CI-Implantation dem Kind möglicherweise die Chance auf ein ``normales'' Leben, einschließlich beruflicher und 
sozialer Möglichkeiten, entzieht. Interessanterweise entschied das Familiengericht am 29. Januar 2019, ``keine familienrechtlichen Maßnahmen'' einzuleiten. 
Begründung:
\begin{itemize}
    \item Nach den Ermittlungen gibt es keine ausreichende Grunde familienrechtliche Maßnahmen anzuordnen
    \item die Kindeseltern können den optimalen Therapieverlauf nach der Implantation nicht gewährleisten
    \item Ohne Aktzeptanz der Eltern ist es unmöglich das das Kind trotz Cochlea Implantat die Hör- und Sprachfähigkeit erlangt
\end{itemize}
~\cite{brde}

\subsection*{Autonomie einer Gruppe}
\begin{itemize}
    \item Wenn Mitglieder einer Minderheit die Möglichkeit haben, ``normal'' zu werden, könnten die Anliegen und Bedürfnisse derjenigen die sich dagegen entscheiden möglicherweise 
    weniger Beachtung finden, da sie nicht mehr als Teil einer diskriminierten Gruppe betrachtet werden. (Argument der leichteren Lösung mach es einfach dann müssen wir nix machen.)
    \item Einzelne Minderheiten und Gruppen haben ihre  eigene kulturelle Dynamik innerhalb ihrer Gemeinschaft diese würde verloren gehen. Beispielsweise die Gehörlosen Communite 
    die in gewisser weise einfach nur Menschen sind die anders komonizieren. Diese andersartigkeit sollte nicht nut akzeptiert sondern auch geschätzt und gefördert werden~\cite{lee2016cochlear}
    \item Eine Betroffene Gruppe könnte jedoch auch an Automomy gewinnen weil sie durch die Technologie wieder Selbstbestimmter durchs Leben gehen kann.~\cite{das2022locked}
\end{itemize}

\subsection*{Unabschätzbare Folgen}
Neue Technologien und Erfindungen bringen oft unvorhersehbare Folgen mit sich, wie in der Vergangenheit bei FCKW die als Kältemittel und als Treibmittel in Spraydosen benutzt wurden 
was zum entstehen des Ozonlochs  geführt hat~\cite{rowland1996stratospheric}. 
Beim Transhumanismus und Posthumanimus würden die Unabschätzbaren folgen jedoch in unserem eigenen Körper auftreten. Vor allem bei Änderungen der DNA kann es fatale und, auch für die 
Nachkommen, irreversible Folgen haben könnte. Die Forschung zum verständnis der DNA ist zwar schon weit jedoch gibt es noch viele ungeklärte Fragen was dazu führen kann das man die 
Änderungen die man an der DNA vornimmt zwar meint zu verstehen jedoch fatale Fehler macht.

Ein  weiteres Beispiel ist Deep Brain Stimulation bei der mithilfe von Elektroden die ins Gehirn implantieren werden elektrische Impulse abgegeben werden die therapeutische Effekte bei 
verschieden neurologischen Erkrankungen haben können. Jedoch hat sich gezeigt, dass die Vorgänge im Gehirn extrem komplex sind und wir weit weg sind sie vollständig verstehen zu können. 
Dadurch können unerwünschte und unerwartete Nebenwirkungen entstehen wie z.B. Depresionen bis hin zum Sozid des Menschen der das DBS implantieren bekommen hat~\cite{zarzycki2020stimulation}.  
Ein weiters Problem ist dass die Elektroden nur großflächig stimulieren können, was zu ungewollten Stimulationen benachbarter Gehirnareale führen kann was wiederum zu Unabschätzbaren Nebenwirkungen
führen kann. Auch die Auswirkungen von kleinen Änderungen der INtensität oder auch die Auswirkungen von kleinen Zeitdelays könnne große Auswirkungen haben~\cite{al2021impact}. In 
Tierversuchen mit Affen zeigte sich, dass eine Frequenzänderung und der Platz der Stimulation des DBS-Implantats den gegenteiligen des gewünschten Effekts hat.~\cite{logothetis2010effects}.

\subsection*{Gesundheit: Heilung und darüber hinaus}
Sollen Maßnahmen zur Steigerung der physischen/psychischen Leistungsfähigkeit erlaubt sein? Im Falle starker Einschränkungen, z.B. Fehlen von Gliedmaßen, Gehörlosigkeit, Tremor bei Parkinson Erkrankten und vielen weiteren
Symptomen verschiedenster Krankheiten ist die Frage anders zu bewerten als bei reiner Steigerung über das „Normallevel“ hinaus. Die Risiken, vor allem das noch eingeschränkte Wissen über den menschlichen Körper und das Gehirn
insbesondere, ist dem Leid der Betroffenen entgegenzustellen. Hierbei nimmt man bestimmte Gefährdungen (Misserfolg, Verletzungen, Tod) in Kauf um den Patienten ein besseres, gesünderes Leben zu ermöglichen. Als Beispiel kann das
Cochlea-Implantat, welches als die erfolgreichste nervenbezogene Prothese (Stand 2011) gilt, aufgeführt werden. Dieses wird tauben Kindern eingesetzt, um sie in der Entwicklung ihres Sprach- und Hörverständnisses zu unterstützen.
Bei den Betroffenen gibt es sowohl sehr erfolgreiche Verläufe, sodass die Kinder nahezu auf dem Level von Gesunden sind aber auch negativere Schicksale, so dass sie deutlich schlechter als gesunde Kinder hören können~\cite{lee2016cochlear}. 
Bei jeder Behandlung besteht also das Risiko, dass das gewünschte Ergebnis nicht oder nur teilweise erreicht wird. Dies gilt auch bei Verfahren und Eingriffen, die rein auf die Verbesserung aus sind. Hierzu kommen wir jetzt.
Den gesundheitlichen Risiken stehen nun nicht mehr der Ausgleich von Einschränkungen, sondern die Ausweitung der menschlichen Leistungsfähigkeit gegenüber. Geforscht wird unter anderem an der Steigerung der kognitiven Fähigkeiten,
wie verbessertes Gedächtnis und bessere Aufmerksamkeit~\cite{SUTHANA2014996}. Aber auch Brain-Computer-Interfaces BCI werden intensiv erforscht. Hierbei soll die Kommunikation zwischen Menschen und Maschine nicht mehr über physische Medien,
also Maus, Tastatur, Bildschirm, stattfinden, sondern direkt über einen Chip im Gehirn. Dieser soll dann auch über fehlende Informationen direkt aus dem Internet einbinden können.~\cite{neuralink} Zur Entscheidung ob solche Technologien verwendet
werden sollen, trägt vor allem auch die Reversibilität der Maßnahmen bei. Besteht diese nicht folgen eine Reihe drastischer Nachteile. Klassische Prothesen für fehlende Gliedmaßen sind reversibel, dass heißt man kann sie anbringen und auch wieder
entfernen, ohne dauerhaft eingeschränkt zu sein. Bei Deep Brain Stimulation, welches unter anderem zur Behandlung der Symptome von Parkinson eingesetzt wird, ist diese Reversibilität nicht mehr gegeben. Bei Schwerkranken ist diese Problematik zu
vernachlässigen, da die Alternative ein menschenunwürdiges Leben beziehungsweise ein baldiger Tod ist. Lässt man aber eine Verbesserung durchführen, kann das dazu führen, dass man zukünftige Behandlungen oder weitere Verbesserungen nicht mehr durchführen kann.

\subsection*{Gesellschaft: Spaltung nimmt zu?}
Technologie zur Verbesserung der körperlichen Leistungsfähigkeit wirft auch einige gesellschaftliche Probleme auf. Vor allem die Finanzierung führt zu etlichen Konflikten. Während im Fall von gesundheitsfördernden Maßnahmen die Krankenkasse als
Instanz der Allgemeinheit Mittel zur Gesundung kranker, eingeschränkter Menschen bereitstellt (in Deutschland, sozialen Marktwirtschaften) werden die Kosten für Verbesserungen von jedem Einzelnen zu tragen sein. Auch heute schon macht die Spaltung der
Gesellschaft aufgrund verschiedener finanzieller Voraussetzungen vor der Gesundheit nicht halt. Während ärmere Menschen mit der gesundheitlichen Grundversorgung der gesetzlichen Krankenkasse auskommen müssen, haben Reiche die Möglichkeit bessere Ärzte, Medikamente,
Behandlungen und gesünderes Essen zu genießen. Dies macht sich auch an der Lebenserwartung deutlich. Wohlhabende Männer leben mit 80,9 Jahren über zehn Jahre länger als ärmere (70,1)~\cite{lampert2014}. Die Möglichkeit die eigene Leistungsfähigkeit und damit
auch Gesundheit zu verbessern (neue Organa, Verjüngung, Mikroroboter), wird finanziell gut Ausgestatteten mehr Möglichkeiten bieten, da sie analog zur Gesundheitsversorgung heute mehr Zugriff auf neue Technologien haben. Die Spaltung der Gesellschaft kann also weiter
zunehmen. Dies führt möglicherweise aber zu größeren Nachteilen, als die Verbesserungen herbeiführen. Besonders kritisch sollte auch die Verschiebung von staatlicher zu immer mehr privater Gesundheitsversorgung beobachtet werden, die aufgrund der Maßnahmen stattfinden
könnte~\cite{khan_aziz_2019}. Auch der Vertrieb der Technologie ist noch nicht geregelt. Sollte der Staat, die Allgemeinheit, solche Eingriffe, Geräte zur Verfügung stellen, stellt sich die Frage, ob die Mittel nicht dafür verwendet werden sollten, Kranke zu heilen, anstatt
Gesunde „besser“ zu machen. Tut er es nicht und Firmen treten an diese Stelle, so steigt die oben beschriebene Ungleichheit zwischen sozio-ökonomischen Klassen~\cite{khan_aziz_2019}. Ein weiteres oft unterschätztes Problem ist die rechtliche Verantwortung bei Eingriffen und
vor allem negativen Folgen dieser. Sollte es den Herstellern solcher Produkte erlaubt sein, die Verantwortung komplett mittels Einverständniserklärung auf die Kunden übertragen zu können? Dies führt eventuell zu riskanteren Eingriffen, da die Firmen nicht haften müssen.
Vergleichbar ist die rechtliche Situation mit dem Haftungsproblem bei autonomen Fahrzeugen. Haftet der Hersteller, der Kunde, der zustimmt die Risiken einzugehen oder die Allgemeinheit.

\subsection*{Umstrittene Akteure und ihre Zukunftsvisionen}
Erwähnenswert ist auch, dass die Forschung an posthumanistischen Ansätzen von Fürsprechern dieser Entwicklung betrieben wird. Als prominentes Beispiel Elon Musk, der mit seiner Firma NeuraLink an BCIs forscht (forschen lässt), die kurzfristig zur Heilung von
Nervenkrankheiten eingesetzt werden sollen. Darüber hinaus wollen sie aber auch Chips für gesunde Menschen entwickeln, die dann zur Steigerung der kognitiven Leistung eingesetzt werden, um beispielsweise direkt im Gehirn mit dem Internet (Computern) zu kommunizieren.
Problematisch ist dabei, dass Elon Musk aber auch häufig in der Kritik steht und nicht unumstritten ist.

\subsection*{Ethische Forschung und wie es aktuell läuft}
Hierbei spielen auch Aspekte der ethischen Forschung eine Rolle. Genau wie bei der Entwicklung von Impfstoff oder Medikamenten (Auch schon Transhumanismus?),
braucht man für Zulassungen von Verbesserungstechnologie menschliche Testteilnehmer. Dies gestaltet sich aber in vielen Ländern als schwierig, da Versuche an Menschen stark reguliert sind und der Zulassung von Gesundheitsbehörden bedürfen.
Um trotzdem teils riskante Eingriffe oder Tests durchführen zu können, setzt man in ärmeren Regionen finanzielle Anreize für die Teilnahme zur Verfügung. Hierbei werden benachteiligte Menschen zu riskanten Eingriffen verleitet, um ihrer misslichen Lage zu entfliehen. 

\section*{Regulierungen - Warum genau jetzt?}
Wie bei vielen neuen Thema in der Forschung ist eine
Auseinandersetzung mit den Folgen und daraus resultierenden, möglichen Einschränkungen auch auf den Transhumansimus bezogen äußerst wichtig. 
So müssen Regulierungen, rechtliche Rahmenbedingungen und Ethikcodizes geschaffen, angewandt und kontrolliert werden, um mögliche Folgen gar nicht erst zuzulassen. \\
Im Hinblick auf Künstliche Intelligenz gab es schon einen größeren, rechtlichen Schritt: Erstellung des Vorschlags für den AI Act der EU 2021~\cite{ai_act_eu_2021},
knapp ein Jahr nach Veröffentlichung von GPT-3 durch OpenAI, und nun einer Vereinbarung zwischen den Mitgliedern des Europäischen Parlaments und Europäischen Rats auf eine genauere Formulierung des Gesetzes~\cite{ai_act_deal_2023}. \\
Somit bleibt die Frage noch offen inwiefern der Transhumanismus jetzt schon besprochen werden sollte, sowie rechtliche und ethische Grundlagen geschaffen werden müssen. Allerdings werden, wie zum Beispiel am Cochlea Implantant zusehen ist,
solche Fragestellungen schon seit einigen Jahren besprochen~\cite{lee2016cochlear}.

\section*{Risiken von Transhumanismus}
// überleitung
\\
Nun stellt sich die Frage nach konkreten möglichen Risiken die mit dem Transhumanismus einhergehen. Diese lassen sich in die drei großen Kategorien Individuum, Organisationen und Gesellschaft eingliedern.\\
Beginnend mit dem Individuum ist die Sicherheit der Technik ein äußerst wichtiges Thema. Alle technischen Erweiterungen haben das Risiko durch dritte gehackt und somit missbraucht zu werden.
Dadurch dass die Geräte nun immer mehr fortschreitenden Einfluss auf die Nutzenden haben, bietet es Hackern auch deutlich mehr Macht im Falle eines Hacks~\cite{khan_aziz_2019}.
Die Geräte können zwar stark geschützt sein, aber durch die möglichen Folgen eines Angriffes sinkt auch das Vertrauen in die Technik selbst~\cite{khan_aziz_2019}.
Auch die Folgen selber eines Angriffes können sehr unscheinbar sein, da beispielesweise die Steuerung eines Gerätes auf so komplizierten Signalen beruht, dass das einwirken Dritter schwer detektierbar ist~\cite{khan_aziz_2019}.
Eine weitere Gefahr geht von den Nutzenden selber aus, da sie sich selbst durch das verändern von Einstellungen der Geräte in riskanten Bereichen gefährden können, mit der Intention sich selber noch weiter zu verbessern~\cite{khan_aziz_2019}.\\
Desweiteren liegt ein Fokus auf der medizinischen Sicherheit nicht nur während der Laufzeit der Geräte, sondern auch bei der Einsetzung und eventuellen Entfernung, im Fokus. So zum Beispiel bestehen Risiken zu irreversiblen Nebeneffekten durch das
Einsetzen und Entfernen eines Brain-Computer-Interfaces wie Infektionen und Traumata des Gehrins~\cite{Burwell:2017aa}. Weitere Risiken bestehen aus dem Eingeschränltem Verständnis des Gehirns~\cite{khan_aziz_2019} und somit unbekannten
Langzeitfolgen im Bezug auf konstante Stimulation und training des Gehirns~\cite{Burwell:2017aa}, Neuronale Plastizität des Gehirns, vor allem bei Kindern,~\cite{Burwell:2017aa} und Fehler in problematischen Situationen, die zu Unfällen
führen können~\cite{Burwell:2017aa}. \\Die nächste Risikokategorie bezieht sich auf Organisationen und Firmen, die durch ihre eigenen Interessen geprägt sind. Dadurch das die Organisationen meist kapitalgetrieben sind, besteht
das Risiko, dass zwischen medizinisch Sinnvollen und Gewinnbringenden Entscheidungen unterschieden wird, und somit nicht im besten Sinne der Nutzenden gehandelt wird~\cite{khan_aziz_2019}. Des weiteren besteht das Risiko des
Datenverkaufs für ``Neuro-Marketing'', welches auch durch die gegebenen, personenbezogen Informationen in die Privatsphäre der Nutzenden eingreifen würde~\cite{khan_aziz_2019}. Noch ein Punkt ist die Gefahr der Monopolbildung.
Durch den ungeregelten Vertrieb der Geräte und die Verschiebung zu privat geregelter Medizinischer Versorgung und damit einhergehender Macht~\cite{khan_aziz_2019}~können einzelne Akteure ihre eigenen Interessen durchsetzen.
Als letzter Punkt sind die möglichen Risiken für die Gesellschaft im Allgemeinen zu betrachten. Beginnend mit dem Verschaffen eines Unfairen Vorteils gegenüber anderen, unter anderem durch die Preishürde, welche unterschiedliche
sozio-ökonomische Klassen noch weiter spalten würde oder auch dem Einsatz zur Vorteilsbeschaffung bei Tests oder im Sport~\cite{khan_aziz_2019}. Auch militärisch könnte es zur Vorteilsgewinnung durch den Einsatz dieser Technik
kommen~\cite{khan_aziz_2019}. Ein weiterer Risikopunkt ist der Verlust der Autonomie einer Gesellschaft und deren Individuen. Somit besteht das Risiko, dass die Menschen zum Werkzeug werden und eine Gesellschaft ihre ``Menschlichkeit''
verliert~\cite{Burwell:2017aa}.


\newpage
\bibliographystyle{IEEEtran}
\bibliography{ref}
\newpage
\end{document}
