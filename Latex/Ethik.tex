% arara: pdflatex: { shell: yes } until !found('log', '\\(?(R|r)e\\)?run (to get|LaTeX)')
\documentclass[a4paper,
DIV=13,
12pt,
BCOR=10mm,
department=FakEI,
%lucida,
%KeepRoman,
twoside,
parskip=half,
automark,
%headsepline,
]{OTHRartcl}

\usepackage[utf8]{inputenc}
\usepackage{breakcites}

%\usepackage[english]{babel}
\usepackage[ngerman]{babel}

\date{\today}

\pagestyle{headings}

\title{Ist Transhumanismus Fortschritt oder Dystopie?}

\author{Marcel Ott}

\documenttype{Ethik}

\begin{document}
\maketitle

\section*{Definitionen bzw. Abgrenzung der Begriffe}

Transhumanismus: Der Transhumanismus beschäftigt sich mit dem Menschen und die Ausschöpfung bzw. Weiterentwicklung seiner natürlichen Grenzen mithilfe von Technik und Wissenschaft~\cite{Merzlyakov2022},
wobei der Mensch als solches weiter beibehalten wird. Beispiele hierfür sind ewiges altern, Prothesen oder Ausmerzen von Krankheiten mittels Hirnimplantaten.

Posthumanimus: Der Posthumanimus sieht den Menschen hingegen als Sackgasse an, welche es zu überwinden gilt. Hierbei werden die binären Gegensetze zwischen Mensch und nicht-Mensch aufgehoben.~\cite{Merzlyakov2022}
Der Cyborg wird oft als nächste Stufe der Evolution betitelt.

Die Grenzen zwischen beiden Begriffen sind jedoch fließend und sie werden oft synonm für einander verwendet, was aber von einigen Forschern kritisiert wird, da es wie gezeigt fundamentale Unterschiede gibt.~\cite{Merzlyakov2022}
Der ein­fach­heits­hal­ber wird in folgendem Bericht mit dem Transhumanismus-Begriff beides eingeschlossen.

Cyborg: Der Begriff wurde das erste mal 1960 von Clynes und Kline verwendet und sie beschrieben damit die technische Anpassung des Menschen an die Bedingungen des Weltalls, anstelle der Verwendung 
einer erdähnlichen Umgebung. Hierbei beziehen sie sich bereits auf die Idee der Evolution, also die Anpassung an die Umweltbedingungen~\cite{clynes1960cyborgs}, ähnlich wie es der Posthumanismus heute.
Prinzipiell beschreibt der Begriff ein integriertes System aus menschlichen und maschinellen Teilen.~\cite{warwick2000cyborg}


\section*{Was ist normal?}
Man kann die Normalität nicht sie klar definieren wie man denkt. Während das intuitive Verständnis den statischen Durchschnitt (in manchen Kontexten macht das auch Sinn oder ist notwendig harte Grenzen zu ziehen z. B. NC bei Studienzulassung)
oder die herrschende Klasse als normal ansieht, ist die Normalität sehr indiviuell und vom Selbst oder Gruppen bestimmt.\cite{schildmann1999normal} Hierbei schaffen sich z. B. Menschen mit Behinderung eigene Communites, wie bei gehörlosen
Personen bei denen man auffällt, wenn man keine Gebärdensprache spricht. Ein Slogan entnommen aus den ethischen Grundaussagen der Lebenshilfe für geistig Behinderte besagt: ´"Es ist normal, verschieden zu sein."´~\cite{lebenshilfeFlyer}

Die Überwachung biotechnologischer Möglichkeiten erfordert zweifellos in erster Linie eine Grenzziehung, d. h. eine Unterscheidung zwischen „therapeutischen“ und „Verbesserungs“-Aktivitäten \cite{breczko2021human}

\section*{Ethische Fragestellungen}

\subsection*{Selbstbestimmung des Indiviuum}
Selbstbestimmung des Indiviuum: Grundätzlich hat jeder das Recht auf freie Entfaltung solange er nicht Rechte anderer oder bestehendes Recht verletzt\cite{fur1996grundgesetz}. Identität kann von Menschen selbst gewählt werden
Argument natürlich bleiben zu wollen. \cite{lee2016cochlear} Von außen schwer zu sagen was Leiden ist deswegen nicht für andere Entscheiden \cite{plavsienkova2021healthy}. Jeder muss für sich selbst abwägen ob er Nebenwirkungen in Kauf nehmen will oder nicht. Dagegen spricht, dass die Entscheidung anderen in gewisser Maßen schadet. Die Gesellschaft muss die Folgen durch Barrierefreiheit, höhere Gesundheitskosten usw. tragen.
Die Frage, welche sich am meisten stellt ist ob bei Menschen, welche nicht selbstbestimmt entscheiden können über den Kopf hinweg entschieden werden darf. Das wohl prägnanteste Beispiel
wäre das Locked-in-Syndrom, bei dem der Hirnstamm beschädigt ist und die betroffenen somit normal denken und fühlen können, jedoch sich nicht bewegen oder sprechen können.\cite{das2022locked}

Ein Cochlea-Implantat ist eine elektronische Vorrichtung, die gehörlosen oder schwerhörigen Menschen hilft, Schallreize wahrzunehmen, indem sie direkt Signale an den Hörnerv sendet.

zusätzlich kann man sich die Frage stellen, ob die Entscheidung der Eltern, ihrem Kind kein Cochlea-Implantat zu implantieren, eine potenzielle Kindswohlgefährdung darstellt, ist von ethischen Überlegungen geprägt
Die HNO-Klinik des Städtischen Klinikums Braunschweig betrachtete die Weigerung der Eltern im Oktober 2017 als Gefährdung des Kindswohls und leitete ein Kinderschutzverfahren ein. Dies führte zu erheblichem Unmut in der Gehörlosengemeinschaft. Die Klinik 
argumentierte, dass die Ablehnung einer CI-Implantation ohne Zustimmung der Eltern dem Kind möglicherweise die Chance auf ein "normales" Leben, einschließlich beruflicher und sozialer Möglichkeiten, entziehen könnte.
Interessanterweise entschied das Familiengericht am 29. Januar 2019, "keine familienrechtlichen Maßnahmen" einzuleiten. Diese Entscheidung reflektiert die Schwierigkeit, eine einheitliche Antwort darauf zu finden, welche Maßnahmen im besten Interesse des Kindes liegen.


\subsection*{Automomy einer Gruppe}
Automomy einer Gruppe: Verbesserung würde Automomy einer Gruppe (einer Minderheit) einschränken. Politische Stimme, eigene Sprache, Einzigartichkeit
\begin{itemize}
    \item Wenn einzelne Mitglieder einer Gruppe die möglichkeit haben "normal" zu werden und somit nicht mehr der Minderheit angehören verliert die Minderheit ihre Politische Stimme und einfluss (Argument gegen die Minderheit die noch bleibt kann sein mach es einfach dann hast du die Probleme nicht mehr)
    \item Einzelne Minderheiten und Gruppen haben ihre eigenen besonderen Sachen die sie ausmacht, könnte dadurch wegfallen.~\cite{lee2016cochlear}
\end{itemize}


\subsection*{Unabschätzbare Folgen}
Neue Technologien bringen oft unvorhersehbare Folgen mit sich, wie in der Vergangenheit bei FCKW, welches das Ozonloch verursacht hat~\cite{rowland1996stratospheric}, gesehen. Nur dieses mal im eigenen Körper.
Vor allem bei Änderungen der DNA kann es fatale und, auch für die Nachkommen, irreversible Folgen haben. Wenn wir das Beispiel von DBS heranziehen hat sich gezeigt, dass die Vorgänge im Gehirn extrem komplex
sind und wir weit weg sind sie noch verstehen zu können. Dadurch können wir den Einfluss bei Änderungen kaum einschätzen (z. B. CFC~\cite{canolty2010functional} oder auch die Auswirkungen von kleinen Zeitdelays~\cite{al2021impact}).
In Tierversuchen mit Affen zeigte sich, dass eine Frequenzänderung und der Platz der Stimulation des DBS-Implantats den gegenteiligen des gewünschten Effekts hat.~\cite{logothetis2010effects}.


\subsection*{Soziale Ungleichheiten}
\begin{itemize}
    \item Schere zwischen arm und reich, da sich das nicht jeder leisten kann
    \item Chance Soziale Ungleichheiten in Bezug auf Minderheiten zu lösen. 
    \item Jedoch nur die Frage ob es sinnvoll ist Soziale Ungleichheiten mithilfe von Technologien zu lösen~\cite{lee2016cochlear} (Frage in Raum stellen oder so kp)
    Da Betroffene dumm angemacht werden warum machen sie das nicht machen somit sich nicht so akzeptiert werden  
\end{itemize}


\subsection*{Umstrittene Akteure}
Ein weiterer bedenklicher Punkt ist, dass zunehmend umstrittene Akteure im Transhumanismus involviert sind. Zunehmend Techmillardäre wie Elon Musk (Neuralink), Peter Thiel und Mark Zuckerberg.
Sie alle zeichnet ein großes Maß an Risikobereitschaft, was im medizinischen Kontext sehr schnell in die Hose gehen kann. Jedoch sind auch andere Größen, wie Kevin Warwick und Julian Savulescu.
Ray Kurzweil der die Singularität anstrebt


\section*{Warum wird der Transhumanismus in Erwägung gezogen?}
Ist diese Optimierung überhaupt nötig oder macht Fehler den Menschen aus
Kapitalistischer Grundgedanke "Verbesserung ist notwendig" und wie man in Fabriken sieht sind uns Maschinen in vielen Bereichen überlegen und der Mensch ist die Schwachstelle

\section*{Warum sollte man sich jetzt schon Gedanken drüber machen?}
AI Act ewig dauert (noch immer nicht beschlossen, da für einige die Regeln zu hart für einige zu weich sind) Es müssen rechtliche Rahmenbedigungen und Ethikkodizes geschaffen werden ~\cite{breczko2021human}
Und wird ja bereits gemacht bzw. Fragestellungen werden bereits seit einigen Jahren besprochen s. Cochlea 2016

\section*{Weitere Gefahren}
Einfach nur nennen und nicht groß drauf eingehen:
\begin{itemize}
    \item Cyberangriffe
    \item Bugs/Fehlverhalten
    \item Fake News direkt ins Gehirn
    \item Abhängigkeit von Produzentenfirmen
    \item Datenmissbrauch
\end{itemize}


\newpage
\bibliographystyle{IEEEtran}
\bibliography{ref}
\newpage
\end{document}