% arara: pdflatex: { shell: yes } until !found('log', '\\(?(R|r)e\\)?run (to get|LaTeX)')
\documentclass[a4paper,
DIV=13,
12pt,
BCOR=10mm,
department=FakEI,
%lucida,
%KeepRoman,
twoside,
parskip=half,
automark,
%headsepline,
]{OTHRartcl}

\usepackage[utf8]{inputenc}
\usepackage{breakcites}

%\usepackage[english]{babel}
\usepackage[ngerman]{babel}

\date{\today}

\pagestyle{headings}

\title{Ist Transhumanismus Fortschritt oder Dystopie?}

\author{Marcel Ott}

\documenttype{Ethik}

\begin{document}
\maketitle

\section*{Definitionen bzw. Abgrenzung der Begriffe}

Transhumanismus: Der Transhumanismus beschäftigt sich mit dem Menschen und die Ausschöpfung bzw. Weiterentwicklung seiner natürlichen Grenzen mithilfe von Technik und Wissenschaft~\cite{Merzlyakov2022},
wobei der Mensch als solches weiter beibehalten wird. Beispiele hierfür sind ewiges altern, Prothesen oder Ausmerzen von Krankheiten mittels Hirnimplantaten.

Posthumanimus: Der Posthumanimus sieht den Menschen hingegen als Sackgasse an, welche es zu überwinden gilt. Hierbei werden die binären Gegensetze zwischen Mensch und nicht-Mensch aufgehoben.~\cite{Merzlyakov2022}
Der Cyborg wird oft als nächste Stufe der Evolution betitelt.

Die Grenzen zwischen beiden Begriffen sind jedoch fließend und sie werden oft synonm für einander verwendet, was aber von einigen Forschern kritisiert wird, da es wie gezeigt fundamentale Unterschiede gibt.~\cite{Merzlyakov2022}
Der ein­fach­heits­hal­ber wird in folgendem Bericht mit dem Transhumanismus-Begriff beides eingeschlossen.

Cyborg: Der Begriff wurde das erste mal 1960 von Clynes und Kline verwendet und sie beschrieben damit die technische Anpassung des Menschen an die Bedingungen des Weltalls, anstelle der Verwendung 
einer erdähnlichen Umgebung. Hierbei beziehen sie sich bereits auf die Idee der Evolution, also die Anpassung an die Umweltbedingungen~\cite{clynes1960cyborgs}, ähnlich wie es der Posthumanismus heute.
Prinzipiell beschreibt der Begriff ein integriertes System aus menschlichen und maschinellen Teilen.~\cite{warwick2000cyborg}

\section*{Warum wird der Transhumanismus in Erwägung gezogen?}
Kapitalistischer Grundgedanke "Verbesserung ist notwendig" und wie man in Fabriken sieht sind uns Maschinen in vielen Bereichen überlegen und der Mensch ist die Schwachstelle

\section*{Aktuelle Technologien}
Bereits heute wird der Transhumanismus intensiv erforscht und auch eingesetzt. Auch wenn noch sehr viele ethische Bedenken und Fragen bestehen. Außerdem sind die Folgen noch sehr schwer abschätzbar
\subsection*{Nanotechnologie}

\subsection*{Anwendungen in der Medizin}
\subsubsection*{Prothesen}
Cochlear Implantat, Hugh Herr https://www.youtube.com/watch?v=PLk8Pm_XBJE "wir werden körperliche Behinderungen im 21. Jahrhundert abschaffen"
agonist-antagonist myoneural interface Prothesen geben kein Feedback zu den Nerven und daher kann man sie nicht Bewegen ohne das man sie sieht
(anspannung sorgt für entspannung des antagonisten und das funktioniert über ein Nervensignal zum Hirn, was ja bei einer Prothese nicht passiert)
mit diesem Interface ist das möglich (benutzt bestimmt ai zur positions predicten, aber noch mal schauen). Also es fühlt sich wenn man die Augen 
verbindet genauso an, wie wenn man seinen natürlichen Fuß bewegt
https://www.media.mit.edu/projects/agonist-antagonist-myoneural-interface-ami/frequently-asked-questions/#:~:text=The%20AMI%20is%20a%20method,is%20stretched%2C%20and%20vice%20versa.
https://www.ncbi.nlm.nih.gov/pmc/articles/PMC8630671/
=> Sowas wird halt sehr teuer sein und somit nicht kaufbar für viele (Schere zwischen Arm und Reich geht noch weiter auseinander)

https://www.youtube.com/watch?v=PLk8Pm_XBJE Robot became part of me 10:15 vllt in Präsentation einblenden? Er fühlt sich nicht wie ein Cyborg
sondern wie ein normaler Mensch => man gewöhnt sich schnell dran

Man könnte sich auch Flügel anbringen
“Once you have tasted flight, you will forever walk the earth with your eyes turned skyward, for there you have been, and there you will always long to return.”
― Leonardo da Vinci
beschreibt die Kraft von Erfahrungen und Erlebnissen. Es sagt aus, dass wenn man einmal den Flug erlebt hat, egal ob buchstäblich im Himmel oder metaphorisch durch persönliche Errungenschaften,
dieser Moment einen für immer prägt. Es weckt ein Verlangen danach, wieder dorthin zurückzukehren, wo man diese Freiheit und Perspektive erfahren hat. Es verdeutlicht, wie tiefgehend bestimmte Erlebnisse uns prägen können und wie sie unsere Träume und Ziele formen können.

neil harbisson hat eine Antenne welche Farben in Geräusche übersetzt und somit kann er farben hören er kann damit auch infrarot und ultraviolette farben "sehen" und hat somit sogar einen Vorteil zum normalen Auge
\subsubsection*{Hirnimplantate}
BCI, DBS
AI in BCI
https://ieeexplore.ieee.org/abstract/document/10185973/

Lock-In-Syndrom und Tetraparese mittels transhumanistischen Mitteln heilen
Irgendeine Random Facharbeit, aber da können wir uns save Literatur und Ideen klauen (ist auch ein Interview mit nem Prof am Schluss)
https://www.liebfrauenschule.de/fileadmin/user_upload/LFSOL/Aktuelles/2021/20210915RozijnFacharbeit.pdf

Neuralink affe der spiele spielen kann nur mit den gedanken (prothesen steuern, aber in zukunft alles andere?)

6ter Finger an einer Hand damit man sachen machen kann die normal 2 hände brauchen => nicht nur ersetzen sondern auch erweitern
https://www.youtube.com/watch?v=fQQEQgugDD8 (2:45)
50 000 Personen haben bereits schon einen Chip in der Hand implantiert (4:15)

atoun Exoskelette zum heben wird bereits in Japan sehr viel benutzt (Firma dahinter will auch sachen wie dritten arm bauen 6:50)
\subsubsection*{Virtual Reality (VR) and Augmented Reality (AR)}

\subsection*{ChatGPT Stuff zu was schon möglich ist}

Neurotechnologie und Brain-Computer Interfaces (BCIs): Fortschritte in der Neurotechnologie haben zu BCIs geführt, die es Menschen ermöglichen, mit Maschinen zu interagieren oder sogar Prothesen direkt mit ihren Gedanken zu steuern. Diese Technologie könnte in Zukunft die Gehirnleistung verbessern oder Menschen mit Behinderungen helfen.

Genbearbeitung (CRISPR): Die CRISPR-Technologie hat die Möglichkeit eröffnet, das menschliche Genom zu verändern. Obwohl ethische Fragen und Sicherheitsbedenken bestehen, wird diese Technik bereits in der Forschung zur Behandlung genetischer Krankheiten eingesetzt.

Nanotechnologie in der Medizin: Nanopartikel werden in der Medizin genutzt, um gezielt Medikamente an bestimmte Stellen im Körper zu transportieren oder für Diagnosezwecke, beispielsweise bei der Früherkennung von Krankheiten.

Prothetik und Exoskelette: Fortschritte in der Robotik ermöglichen immer fortschrittlichere Prothesen und Exoskelette, die die Mobilität und Fähigkeiten von Menschen mit Amputationen oder Behinderungen verbessern.

Kryonik und Lebensverlängerungsforschung: Kryonikunternehmen bieten die Möglichkeit an, Körper nach dem Tod einzufrieren, in der Hoffnung auf zukünftige Technologien, die diese wiederbeleben könnten. Außerdem wird intensiv an Methoden zur Lebensverlängerung und Bekämpfung des Alterns geforscht.

Augmented Reality (AR) und Virtual Reality (VR): Obwohl sie derzeit eher im Unterhaltungsbereich verbreitet sind, haben AR und VR Potenzial für transhumanistische Anwendungen. Sie könnten in der Zukunft für Bildung, Simulationen oder sogar für die Erweiterung der menschlichen Wahrnehmung eingesetzt werden.

Biohacking und DIY-Implantate: Die Biohacking-Bewegung setzt auf DIY-Biologie und -Technologie, um die eigenen Fähigkeiten zu erweitern. Das reicht von implantierten Sensoren bis hin zu DIY-Genschneidetechniken, die nicht unbedingt von etablierten wissenschaftlichen Institutionen stammen.

Roboterassistenten und KI-basierte Pflege: In der Pflegeindustrie gewinnen Roboter und KI-gestützte Systeme an Bedeutung. Sie könnten in der Zukunft eine größere Rolle bei der Unterstützung älterer oder behinderter Menschen spielen.

Experimentelle Gen- und Zelltherapien: Fortschritte in der Gen- und Zelltherapie könnten die Behandlung von Krankheiten revolutionieren, indem sie direkt auf der genetischen Ebene intervenieren, um Krankheiten zu heilen oder zu verhindern.


\section*{ChatGPT Stuff zu Akzeptanz von Transhumanismus zwischen verschiedenen Gruppen}
Altersgruppen: Jüngere Generationen neigen häufiger dazu, transhumanistische Ideen zu akzeptieren oder zumindest offener dafür zu sein. Sie sind oft technologieaffiner und offener für neue wissenschaftliche Entwicklungen, die das Potenzial haben, die menschliche Existenz zu verändern. Ältere Generationen können skeptischer sein, da sie möglicherweise stärker an traditionelle Ansichten über Menschlichkeit und Ethik gebunden sind.

Länder und Kulturen: Die Akzeptanz von Transhumanismus variiert stark je nach kulturellem Hintergrund und den Werten eines Landes. In einigen Ländern, die technologische Innovationen stark fördern, gibt es möglicherweise eine größere Akzeptanz für transhumanistische Ideen. Länder mit starken religiösen oder ethischen Überzeugungen könnten jedoch aufgrund moralischer Bedenken oder ethischer Grundsätze weniger offen für solche Konzepte sein.

Ethik und Wertvorstellungen: Ethische Überlegungen spielen eine entscheidende Rolle bei der Akzeptanz des Transhumanismus. Einige befürworten diese Ideen als Möglichkeit, menschliche Leiden zu verringern oder die menschliche Lebensqualität zu verbessern, während andere sie als gefährlich oder ethisch bedenklich betrachten, insbesondere wenn es um die Manipulation des menschlichen Genoms oder die Schaffung von Künstlicher Intelligenz geht.

Bildung und Zugang zur Information: Menschen mit einem tieferen Verständnis für die zugrunde liegende Technologie und deren potenzielle Auswirkungen sind möglicherweise eher bereit, transhumanistische Ideen zu akzeptieren. Der Zugang zu Bildung und Informationen spielt hierbei eine bedeutende Rolle.

\section*{Zweifelhafte Akteure}
Bezogen auf Personen, wie Elon Musk, Julian Savulescu, Kevin Warwick

\section*{Fazit}
Technik hat generell keine Moral und kann für gute als auch schlechte Sachen genutzt werden.
\newpage
\bibliographystyle{apalike}
\bibliography{ref}
\newpage
\end{document}